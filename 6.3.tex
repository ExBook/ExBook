\subsection{图的遍历}\qanswerloc{228}
\begin{qitems}
    \begin{bbox}
        \qitem 图 $G=(V,E)$ 以邻接表存储,如下图所示,试画出图 G 的深度优先生成树和广度优先生成树(假设从结点 1 开始遍历)。
        \imgin[0.8]{}{fig/fig6.3.1.png}
    \end{bbox}

    \begin{bbox}
        \qitem 给定一个连通无向图,将图的所有顶点分别染成红色或蓝色,若存在一种染色方法使图中每条边的两个顶点的颜色都不同,则称这个图能被二分。对于下图所示的两个无向图:
        \imgin[0.6]{}{fig/fig6.3.2.png}
        \begin{subqitems}
            \subqitem 判断上面两个无向图是否能被二分,若能二分,则请标出每个顶点的颜色。
            \subqitem 请设计一种算法来判断是否能被二分,仅用语言描述算法的思想即可。
            \subqitem 给出你设计的算法的时间复杂度和空间复杂度。
        \end{subqitems}
    \end{bbox}
    \begin{bbox}
        \qitem 试设计一个算法,判断一个无向图 G 是否为一棵树。若是,则算法返回 true,否则返回 false。
    \end{bbox}

    \begin{bbox}
        \qitem 分别采用基于深度优先遍历和广度优先遍历算法判别以邻接表方式存储的有向图中是否存在由顶点 $v_i$ 到顶点 $v_j$ 的路径($i \neq j$)。注意,算法中涉及的图的基本操作必须在此存储结构上实现。
    \end{bbox}

    \begin{bbox}
        \qitem 假设图用邻接表表示,设计一个算法,输出从顶点 $V_i$ 到顶点 $V_j$ 的所有简单路径。
    \end{bbox}
\end{qitems}