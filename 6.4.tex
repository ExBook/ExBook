\subsection{图的应用}\qanswerloc{259}
\begin{qitems}
    \begin{bbox}
        \qitem 下面是一种称为"破圈法"的求解最小生成树的方法:所谓"破圈法",是指"任取一圈,去掉圈上权最大的边",反复执行这一步骤,直到没有圈为止。试判断这种方法是否正确。若正确,说明理由;若不正确,举出反例(注:圈就是回路)。
    \end{bbox}
    \begin{bbox}
        \qitem 已知有向图如下图所示。
        \imgin[0.7]{}{fig/fig6.4.2.png}
        \begin{subqitems}
            \subqitem 写出该图的邻接矩阵表示并据此给出从顶点 1 出发的深度优先遍历序列。
            \subqitem 求该有向图的强连通分量的数目。
            \subqitem 给出该图的任意两个拓扑序列。
            \subqitem 若将该图视为无向图,分别用 Prim 算法和 Kruskal 算法求最小生成树。
        \end{subqitems}
    \end{bbox}
    \begin{bbox}
        \qitem 对下图所示的无向图,按照 Dijkstra 算法,写出从顶点 1 到其他各个顶点的最短路径和最短路径长度(顺序不能颠倒)。
        \imgin[0.8]{}{fig/fig6.4.3.png}
    \end{bbox}
    \begin{bbox}
        \qitem 下图所示为一个用 AOE 网表示的工程。
        \imgin[0.6]{}{fig/fig6.4.4.png}
        \begin{subqitems}
            \subqitem 画出此图的邻接表表示。
            \subqitem 完成此工程至少需要多少时间?
            \subqitem 指出关键路径。
            \subqitem 哪些活动加速可以缩短完成工程所需的时间?
        \end{subqitems}
    \end{bbox}
    \begin{bbox}
        \qitem 下表给出了某工程各工序之间的优先关系和各工序所需的时间(其中"—"表示无先驱工序),请完成以下各题:
        \begin{center}
        \begin{tabular}{|l|c|c|c|c|c|c|c|c|}
        \hline
        工序代号 & A & B & C & D & E & F & G & H \\
        \hline
        所需时间 & 3 & 2 & 2 & 3 & 4 & 3 & 2 & 1 \\
        \hline
        先驱工序 & --- & --- & A & A & B & A & C, E & D \\
        \hline
        \end{tabular}
        \end{center}
        \begin{subqitems}
            \subqitem 画出相应的 AOE 网。
            \subqitem 列出各事件的最早发生时间和最迟发生时间。
            \subqitem 求出关键路径并指明完成该工程所需的最短时间。
        \end{subqitems}
    \end{bbox}
    \begin{bbox}
        \qitem 试编写利用 DFS 实现有向无环图拓扑排序的算法。
    \end{bbox}
    \begin{bbox}
        \qitem 【2009 统考真题】带权图(权值非负,表示边连接的两顶点间的距离)的最短路径问题是找出从初始顶点到目标顶点之间的一条最短路径。假设从初始顶点到目标顶点之间存在路径,现有一种解决该问题的方法:\\
        \ding{172} 设最短路径初始时仅包含初始顶点,令当前顶点 $u$ 为初始顶点。\\
        \ding{173} 选择离 $u$ 最近且尚未在最短路径中的一个顶点 $v$,加入最短路径,修改当前顶点 $u=v$。\\
        \ding{174} 重复步骤\ding{173},直到 $u$ 是目标顶点时为止。\\
        请问上述方法能否求得最短路径?若该方法可行,请证明;否则,请举例说明。
    \end{bbox}
    \begin{bbox}
        \qitem 【2011 统考真题】已知有 6 个顶点(顶点编号为 0$\sim$5)的有向带权图 G,其邻接矩阵 A 为上三角矩阵,按行为主序(行优先)保存在如下的一维数组中。
        \begin{center}
        \begin{tabular}{|c|c|c|c|c|c|c|c|c|c|c|c|c|c|c|}
        \hline
        4 & 6 & $\infty$ & $\infty$ & $\infty$ & 5 & $\infty$ & $\infty$ & $\infty$ & 4 & 3 & $\infty$ & $\infty$ & 3 & 3 \\
        \hline
        \end{tabular}
        \end{center}
        要求:
        \begin{subqitems}
            \subqitem 写出图 G 的邻接矩阵 A。
            \subqitem 画出有向带权图 G。
            \subqitem 求图 G 的关键路径,并计算该关键路径的长度。
        \end{subqitems}
    \end{bbox}
    \begin{bbox}
        \qitem 【2014 统考真题】某网络中的路由器运行 OSPF 路由协议,下表是由路由器 R1 维护的主要链路状态信息(LSI),R1 构造的网络拓扑图(见下图)是根据题下表及 R1 的接口名构造出来的网络拓扑。
        \imgin[0.6]{}{fig/fig6.4.10.png}
        请回答下列问题。
        \begin{subqitems}
            \subqitem 本题中的网络可抽象为数据结构中的哪种逻辑结构?
            \subqitem 针对表中的内容,设计合理的链式存储结构,以保存表中的链路状态信息(LSI)。要求给出链式存储结构的数据类型定义,并画出对应表的链式存储结构示意图(示意图中可以仅以 ID 标识结点)。
            \subqitem 按照 Dijkstra 算法的策略,依次给出 R1 到达子网 192.1.x.x 的最短路径及费用。
        \end{subqitems}
    \end{bbox}
    \begin{bbox}
        \qitem 【2017 统考真题】使用 Prim 算法求带权连通图的最小(代价)生成树(MST)。请回答下列问题:
        \imgin[0.7]{}{fig/fig6.4.11.png}
        \begin{subqitems}
            \subqitem 对下列图 G,从顶点 A 开始求 G 的 MST,依次给出按算法选出的边。
            \subqitem 图 G 的 MST 是唯一的吗?
            \subqitem 对任意的带权连通图,满足什么条件时,其 MST 是唯一的?
        \end{subqitems}
    \end{bbox}
    \begin{bbox}
        \qitem 【2018 统考真题】拟建设一个光通信骨干网络连通 BJ、CS、XA、QD、JN、NJ、TL 和 WH 等 8 个城市,下图中无向边上的权值表示两个城市之间备选光缆的铺设费用。
        \imgin[0.7]{}{fig/fig6.4.12.png}
        请回答下列问题:
        \begin{subqitems}
            \subqitem 仅从铺设费用角度出发,给出所有可能的最经济的光缆铺设方案(用带权图表示),并计算相应方案的总费用。
            \subqitem 该图可采用图的哪种存储结构?给出求解问题 1)所用的算法名称。
            \subqitem 假设每个城市采用一个路由器按 1)中得到的最经济方案组网,主机 H1 直接连接 TL 的路由器,主机 H2 直接连接 BJ 的路由器。若 H1 向 H2 发送一个 TTL = 5 的 IP 分组,则 H2 是否可以收到该 IP 分组?
        \end{subqitems}
    \end{bbox}
    \begin{bbox}
        \qitem 【2024 统考真题】2023 年 10 月 26 日,神舟十七号载人飞船发射取得圆满成功,再次彰显了中国航天事业的辉煌成就。载人航天工程是包含众多子工程的复杂系统工程,为了保证工程的有序开展,需要明确各子工程的前导子工程,以协调各子工程的实施。该问题可以简化、抽象为有向图的拓扑序列问题。已知有向图 G 采用邻接矩阵存储,类型定义如下。
        \begin{lstlisting}[language=C, basicstyle=\ttfamily\small]
typedef struct {                //图的类型定义
    int numVertices, numEdges;  //图的顶点数和有向边数
    char VerticesList[MAXV];    //顶点表, MAXV为已定义常量
    int Edge[MAXV][MAXV];       //邻接矩阵
} MGraph;
        \end{lstlisting}
        请设计算法:\lstinline{int uniquely(MGraph G)},判定 G 是否存在唯一的拓扑序列,若是,则返回 1,否则返回 0。要求如下。
        \begin{subqitems}
            \subqitem 给出算法的基本设计思想。
            \subqitem 根据设计思想,采用 C 或 C++语言描述算法,关键之处给出注释。
        \end{subqitems}
    \end{bbox}
\end{qitems} 