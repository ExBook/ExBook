\subsection{B树和B+树}\qanswerloc{321}

\begin{qitems}
    \begin{bbox}
        \qitem 给定一组关键字 \{20, 30, 50, 52, 60, 68, 70\},给出创建一棵 3 阶 B 树的过程。
    \end{bbox}
    \begin{bbox}
        \qitem 对如下图所示的 3 阶 B 树,依次执行下列操作,画出各步操作的结果。
        \begin{center}
            \begin{tikzpicture}[level distance=2cm,
                level 1/.style={draw, rectangle, sibling distance=4cm, inner sep=5pt},
                level 2/.style={draw, rectangle, sibling distance=2cm, inner sep=5pt}]
                \node {50}
                    child { node {30}
                        child { node {8, 20} }
                        child { node {35, 40} }
                    }
                    child { node {80}
                        child { node {60} }
                        child { node {100} }
                    };
            \end{tikzpicture}
        \end{center}
        \begin{subqitems}
            \subqitem 插入 90
            \subqitem 插入 25
            \subqitem 插入 45
            \subqitem 删除 60
            \subqitem 删除 80
        \end{subqitems}
    \end{bbox}
    \begin{bbox}
        \qitem 利用 B 树做文件索引时,若假设磁盘页块的大小是 4000B (实际应是 2 的次幂,此处是为了计算方便),
        指示磁盘地址的指针需要 5B。现有 20000000 个记录构成的文件,每个记录为 200B,其中包括关键字 5B。
        试问在这个采用 B 树作索引的文件中,B 树的阶数应为多少?
        假定文件数据部分未按关键字有序排列,则索引部分需要占用多少磁盘页块?
    \end{bbox}
\end{qitems} 