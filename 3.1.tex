\subsection{栈}\qanswerloc{73}

\begin{qitems}

    \begin{bbox}
        \qitem 有 5 个元素,其入栈次序为 A, B, C, D, E, 在各种可能的出栈次序中,第一个出栈元素为 C 且第二个出栈元素为 D 的出栈序列有哪几个?
    \end{bbox}

    \begin{bbox}
        \qitem 若元素的入栈序列为 A, B, C, D, E, 运用栈操作,能否得到出栈序列 B, C, A, E, D 和 D, B, A, C, E?为什么?
    \end{bbox}

    \begin{bbox}
        \qitem 栈的初态和终态均为空,以 I 和 O 分别表示入栈和出栈,则出入栈的操作序列可表示为由 I 和 O 组成的序列,可以操作的序列称为合法序列,否则称为非法序列。
        \begin{subqitems}
            \subqitem 下面所示的序列中哪些是合法的?
            
            A. IOIIOIOO \quad B. IOOIOIIO \quad C. IIIOIOIO \quad D. IIIOOIOO
            \subqitem 通过对 1) 的分析,写出一个算法,判定所给的操作序列是否合法。若合法,返回 \texttt{true}, 否则返回 \texttt{false} (假定被判定的操作序列已存入一维数组中)。
        \end{subqitems}
    \end{bbox}

    \begin{bbox}
        \qitem 设单链表的表头指针为 L, 结点结构由 data 和 next 两个域构成,其中 data 域为字符型。试设计算法判断该链表的全部 n 个字符是否中心对称。例如 xyx、xyyx 都是中心对称。
    \end{bbox}

    \begin{bbox}
        \qitem 设有两个栈 $S_1$、$S_2$ 都采用顺序栈方式,并共享一个存储区 $[0, \dots, \text{maxsize}-1]$,为了尽量利用空间,减少溢出的可能,可采用栈顶相向、迎面增长的存储方式。试设计 $S_1$、$S_2$ 有关入栈和出栈的操作算法。
    \end{bbox}

\end{qitems}
