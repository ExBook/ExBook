\subsection{树与二叉树的应用}\qanswerloc{194}

\begin{qitems}
    \begin{bbox}
        \qitem 设给定权集 $w = \{5, 7, 2, 3, 6, 8, 9\}$,试构造关于 $w$ 的一棵哈夫曼树,并求其加权路径长度 WPL。
    \end{bbox}

    \begin{bbox}
        \qitem 【2012 统考真题】设有 6 个有序表 A, B, C, D, E, F,分别含有 10, 35, 40, 50, 60 和 200 个数据元素,各表中的元素按升序排列。要求通过 5 次两两合并,将 6 个表最终合并为 1 个升序表,并使最坏情况下比较的总次数达到最小。请回答下列问题:
        \begin{subqitems}
            \subqitem 给出完整的合并过程,并求出最坏情况下比较的总次数。
            \subqitem 根据你的合并过程,描述 $n(n \ge 2)$ 个不等长升序表的合并策略,并说明理由。
        \end{subqitems}
    \end{bbox}

    \begin{bbox}
        \qitem 【2020 统考真题】若任意一个字符的编码都不是其他字符编码的前缀,则称这种编码具有前缀特性。现有某字符集(字符个数 $\ge 2$)的不等长编码,每个字符的编码均为二进制的 0、1 序列,最长为 $L$ 位,且具有前缀特性。请回答下列问题:
        \begin{subqitems}
            \subqitem 哪种数据结构适宜保存上述具有前缀特性的不等长编码?
            \subqitem 基于你所设计的数据结构,简述从 0/1 串到字符串的译码过程。
            \subqitem 简述判定某字符集的不等长编码是否具有前缀特性的过程。
        \end{subqitems}
    \end{bbox}

\end{qitems} 