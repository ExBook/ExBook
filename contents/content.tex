\clearpage
\clearpage
\section{绪论}
\subsection{数据结构的基本概念}
% 答案见原书P4
\qnum \quad \qanswerloc{4}

\begin{questions}[tr]
    \begin{bbox}
        \question[2] 可以用\blankbox 定义一个完整的数据结构。
        \fourchoices{数据元素}{数据对象}{数据关系}{抽象数据类型}     
    \end{bbox}

    \begin{bbox}
        \question[2] 以下数据结构中,\blankbox 是非线性数据结构。
        \fourchoices{树}{字符串}{队列}{栈}
    \end{bbox}

    \begin{bbox}
 
        \question[2] 以下属于逻辑结构的是 \blankbox 。
        \fourchoices{顺序表}{哈希表}{有序表}{单链表}
        
    \end{bbox}

    % \begin{bbox}
 
    %     \question[2] 以下与数据的存储结构无关的术语是 \blankbox 。
    %     \fourchoices{循环队列}{链表}{哈希表}{栈}
    % \end{bbox}

    \begin{bbox}
        \question[2] 以下关于数据结构的说法中,正确的是\blankbox 。
        \fourchoices{数据的逻辑结构独立于其存储结构}{数据的存储结构独立于其逻辑结构}{数据的逻辑结构唯一决定 其存储结构}{数据结构仅由其逻辑结构和存储结构决定}
    \end{bbox}

    \begin{bbox}
        \question[2] 在存储数据时,通常不仅要存储各数据元素的值,而且要存储\blankbox
        \fourchoices{数据的操作方法}{数据元素的类型}{数据元素之间的关系}{数据的存取方法}
    \end{bbox}

    \begin{bbox}
        \question[2] 分析以下各程序段, 求出算法的时间复杂度.
        \begin{lstlisting}[escapeinside={(*@}{@*)}]
        (*@\ding{172}:@*)
        i=1; k=0;
        while(i<n-1){
            k=k+10*i;
            i++;
        }
    
        (*@\ding{173}:@*)
        y=0;
        while((y+1)*(y+1)<=n)
        y=y+1;
    
        (*@\ding{174}:@*)
        for(i=0;i<n;i++)
            for(j=0;j<m;j++)
                a[i][j]=0;
    
        
        \end{lstlisting}
    \end{bbox}

    \begin{bbox}
        \question[2] 【2011 统考真题】一个长度为 $L$($L\geqslant 1$ )的升序序列$ S$, 处在第$\lceil L/2\rceil $个位置的数称为 $S$
        的中位数。例如,若序列 $S_1$=(11,13,15,17,19), 则 $S_1$的中位数是 15, 两 个序列的中位
        数是含它们所有元素的升序序列的中位数。例如,若 $S_2$ =(2,4,6,8,20), 则$S_1$和$S_2$的中
        位数是 11。现在有两 个等长升序序列$A$和$B$, 试设计一个在时间和空间两 方面都尽可能
        高效的算法,找出两个序列 $A$和$B$的中位数。要求:
        \begin{subquestions}
            \subquestion 给出算法的基本设计思想
            \subquestion  根据设计思想,采用 C 或 C++或 Java 语言描述算法,关键之处给出注释
            \subquestion 说明你所设计算法的时间复杂度和空间复杂度
        \end{subquestions}
    \end{bbox}
\end{questions}
    