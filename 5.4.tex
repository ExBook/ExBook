\subsection{树、森林}\qanswerloc{181}

\begin{qitems}
    \begin{bbox}
        \qitem 给定一棵树的先根遍历序列和后根遍历序列,能否唯一确定一棵树?若能,请举例说明;若不能,请给出反例。
    \end{bbox}

    \begin{bbox}
        \qitem 将下面一个由 3 棵树组成的森林转换为二叉树。
        \begin{center}
            \begin{forest}
                for tree={circle, draw, l=1.5cm},
                [A,
                    [B]
                    [C]
                ]
            \end{forest}
            \hspace{1cm}
            \begin{forest}
                for tree={circle, draw},
                [D,
                    [E,
                        [F]
                    ]
                ]
            \end{forest}
            \hspace{1cm}
            \begin{forest}
                for tree={circle, draw, l=1.5cm},
                [G,
                    [H,
                        [K]
                        [L]
                    ]
                    [I]
                    [J,
                        [M,
                            [P]
                        ]
                        [N]
                        [O]
                    ]
                ]
            \end{forest}
        \end{center}
    \end{bbox}

    \begin{bbox}
        \qitem 已知某二叉树的先序序列和中序序列分别为 \texttt{ABDEHCFIMGJKL} 和 \texttt{DBHEAIMFGCKLJ},请画出这棵二叉树,并画出二叉树对应的森林。
    \end{bbox}

    \begin{bbox}
        \qitem 编程求以孩子兄弟表示法存储的森林的叶结点数。
    \end{bbox}

    \begin{bbox}
        \qitem 以孩子兄弟链表为存储结构,请设计递归算法求树的深度。
    \end{bbox}

\end{qitems} 