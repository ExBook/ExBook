\subsection{外部排序}\qanswerloc{396}

\begin{qitems}
    \begin{bbox}
        \qitem 若某个文件经内部排序得到 80 个初始归并段,试问:
        \begin{subqitems}
            \subqitem 若使用多路平衡归并并执行 3 趟完成排序,则应取得的归并路数至少应为多少?
            \subqitem 若操作系统要求一个程序同时可用的输入/输出文件的总数不超过 15 个,则按多路归并至少需要几趟可以完成排序?若限定这个趟数,可取的最低路数是多少?
        \end{subqitems}
    \end{bbox}
    \begin{bbox}
        \qitem 假设文件有 4500 个记录,在磁盘上每个块可放 75 个记录。计算机中用于排序的内存区可容纳 450 个记录。试问:
        \begin{subqitems}
            \subqitem 可以建立多少个初始归并段?每个初始归并段有多少记录?存放于多少个块中?
            \subqitem 应采用几路归并?请写出归并过程及每趟需要读/写磁盘的块数。
        \end{subqitems}
    \end{bbox}
    \begin{bbox}
        \qitem 设初始归并段为(10, 15, 31), (9, 20), (22, 34, 37), (6, 15, 42), (12, 37), (84, 95)。试利用败者树进行 m 路归并,手工执行选择最小的 5 个关键字的过程。
    \end{bbox}
    \begin{bbox}
        \qitem 给出 12 个初始归并段,其长度分别为 30, 44, 8, 6, 3, 20, 60, 18, 9, 62, 68, 85。现要做 4 路外归并排序,试画出表示归并过程的最佳归并树,并计算该归并树的带权路径长度 WPL。
    \end{bbox}
    \begin{bbox}
        \qitem 【2023 统考真题】对含有 $n$ ($n>0$) 个记录的文件进行外部排序,采用置换-选择排序生成初始归并段时需要使用一个工作区,工作区中能保存 $m$ 个记录。请回答:
        \begin{subqitems}
            \subqitem 若文件中含有 19 个记录,其关键字依次是 51, 94, 37, 92, 14, 63, 15, 99, 48, 56, 23, 60, 31, 17, 43, 8, 90, 166, 100,则当 $m=4$ 时,可生成几个初始归并段?各是什么?
            \subqitem 对任意的 $m$ ($n \gg m > 0$),生成的第一个初始归并段的长度最大值和最小值分别是多少?
        \end{subqitems}
    \end{bbox}
\end{qitems} 