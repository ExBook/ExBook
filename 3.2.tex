\subsection{队列}\qanswerloc{87}

\begin{qitems}

    \begin{bbox}
        \qitem 若希望循环队列中的元素都能得到利用,则需设置一个标志域 \texttt{tag},并以 \texttt{tag} 的值为 0 或 1 来区分队首指针 \texttt{front} 和队尾指针 \texttt{rear} 相同时的队列状态是"空"还是"满"。试编写与此结构相应的入队和出队算法。
    \end{bbox}

    \begin{bbox}
        \qitem Q 是一个队列,S 是一个空栈,实现将队列中的元素逆置的算法。
    \end{bbox}

    \begin{bbox}
        \qitem 利用两个栈 S1 和 S2 来模拟一个队列,已知栈的 4 个运算定义如下:
        \begin{lstlisting}[language=C, basicstyle=\ttfamily\small]
Push(S,x);        // 元素 x 入栈 S
Pop(S,x);         // S 出栈并将出栈的值赋给 x
StackEmpty(S);    // 判断栈是否为空
StackOverflow(S); // 判断栈是否为满
        \end{lstlisting}
        如何利用栈的运算来实现该队列的 3 个运算 (形参由读者根据要求自己设计)?
        \begin{lstlisting}[language=C, basicstyle=\ttfamily\small]
Enqueue(x);       // 将元素 x 入队
Dequeue(x);       // 出队,并将出队元素存储在 x 中
QueueEmpty();     // 判断队列是否为空
        \end{lstlisting}
    \end{bbox}

    \begin{bbox}
        \qitem 【2019 统考真题】请设计一个队列,要求满足:1.初始时队列为空;2.入队时,允许增加队列占用空间;3.出队后,出队元素所占用的空间可重复使用,即整个队列所占用的空间只增不减;4.入队操作和出队操作的时间复杂度始终保持为 O(1)。请回答:
        \begin{subqitems}
            \subqitem 该队列是应选择链式存储结构,还是应选择顺序存储结构?
            \subqitem 画出队列的初始状态,并给出判断队空和队满的条件。
            \subqitem 画出第一个元素入队后的队列状态。
            \subqitem 给出入队操作和出队操作的基本过程。
        \end{subqitems}
    \end{bbox}

\end{qitems} 