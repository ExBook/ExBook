\subsection{线性表的顺序表示}\qanswerloc{21}

\begin{qitems}

    \begin{bbox}
        \qitem 从顺序表中删除具有最小值的元素 (假设唯一) 并由函数返回被删元素的值。空出的位置由最后一个元素填补,若顺序表为空,则显示出错信息并退出运行。
    \end{bbox}

    \begin{bbox}
        \qitem 设计一个高效算法,将顺序表 L 的所有元素逆置,要求算法的空间复杂度为 O(1)。
    \end{bbox}

    \begin{bbox}
        \qitem 对长度为 $n$ 的顺序表 $L$, 编写一个时间复杂度为 $O(n)$、空间复杂度为 $O(1)$ 的算法,该算法删除顺序表中所有值为 $x$ 的元素。
    \end{bbox}

    \begin{bbox}
        \qitem 从顺序表中删除其值在给定值 $s$ 和 $t$ 之间 (包含 $s$ 和 $t$, 要求 $s<t$) 的所有元素,若 $s$ 或 $t$ 不合理或顺序表为空,则显示出错信息并退出运行。
    \end{bbox}

    \begin{bbox}
        \qitem 从有序顺序表中删除所有其值重复的元素,使表中所有元素的值均不同。
    \end{bbox}

    \begin{bbox}
        \qitem 将两个有序顺序表合并为一个新的有序顺序表,并由函数返回结果顺序表。
    \end{bbox}

    \begin{bbox}
        \qitem 已知在二维数组 $A[m+n]$ 中依次存放两个线性表 $(a_1,a_2,a_3,\dots,a_m)$ 和 $(b_1,b_2,b_3,\dots,b_n)$。编写一个函数,将数组中两个顺序表的位置互换,即将 $(b_1,b_2,b_3,\dots,b_n)$ 放在 $(a_1,a_2,a_3,\dots,a_m)$ 的前面。
    \end{bbox}

    \begin{bbox}
        \qitem 线性表 $(a_1,a_2,a_3,\dots,a_n)$ 中的元素递增有序且按顺序存储于计算机内。要求设计一个算法,完成用最少时间在表中查找数值为 $x$ 的元素,若找到,则将其与后继元素位置相交换,若找不到,则将其插入表中并使表中元素仍递增有序。
    \end{bbox}

    \begin{bbox}
        \qitem 给定三个序列 $A、B、C$,长度均为 $n$, 且均为无重复元素的递增序列,请设计一个时间上尽可能高效的算法,逐行输出同时存在于这三个序列中的所有元素。例如,数组 $A$ 为 $\{1,2,3\}$, 数组 $B$ 为 $\{2,3,4\}$, 数组 $C$ 为 $\{-1,0,2\}$, 则输出 $2$。要求:
        \begin{subqitems}
            \subqitem 给出算法的基本设计思想。
            \subqitem 根据设计思想,采用 C 或 C++语言描述算法,关键之处给出注释。
            \subqitem 说明你的算法的时间复杂度和空间复杂度。
        \end{subqitems}
    \end{bbox}

    \begin{bbox}
        \qitem 【2010 统考真题】设将 $n$ ($n>1$) 个整数存放到一维数组 $R$ 中。设计一个在时间和空间
        两方面都可能高效的算法,将 $R$ 中保存的序列循环左移 $P$ ($0<P<n$) 个位置,即将 $R$
        中的数据 $(X_0, X_1, \dots, X_{n-1})$ 变换为 $(X_P, X_{P+1}, \dots, X_{n-1}, X_0, X_1, \dots, X_{P-1})$。要求:
        \begin{subqitems}
            \subqitem 给出算法的基本设计思想。
            \subqitem 根据设计思想,采用 C 或 C++语言描述算法,关键之处给出注释。
            \subqitem 说明你所设计算法的时间复杂度和空间复杂度。
        \end{subqitems}
    \end{bbox}

    \begin{bbox}
        \qitem 【2011 统考真题】一个长度为 $L$($L\geqslant 1$ )的升序序列$ S$, 处在第$\lceil L/2\rceil $个位置的数称为 $S$
        的中位数。例如,若序列 $S_1$=(11,13,15,17,19), 则 $S_1$的中位数是 15, 两个序列的中位
        数是含它们所有元素的升序序列的中位数。例如,若 $S_2$ =(2,4,6,8,20), 则$S_1$和$S_2$的中
        位数是 11。现在有两 个等长升序序列$A$和$B$, 试设计一个在时间和空间两方面都尽可能
        高效的算法,找出两个序列 $A$和$B$的中位数。要求:
        \begin{subqitems}
            \subqitem 给出算法的基本设计思想
            \subqitem  根据设计思想,采用 C 或 C++或 Java 语言描述算法,关键之处给出注释
            \subqitem 说明你所设计算法的时间复杂度和空间复杂度
        \end{subqitems}
    \end{bbox}

    \begin{bbox}
        \qitem 【2013 统考真题】已知一个整数序列 $A=(a_0, a_1, \dots, a_{n-1})$,其中 $0 \le a_i < n$ ($0 \le i < n$)。
        若存在 $a_{p_1} = a_{p_2} = \dots = a_{p_m} = x$ 且 $m > n/2$ ($0 \le p_k < n, 1 \le k \le m$),
        则称 $x$ 为 $A$ 的主元素。例如 $A=(0, 5, 5, 3, 5, 7, 5, 5)$,则 $5$ 为主元素;
        又如 $A=(0, 5, 5, 3, 5, 1, 5, 7)$,则 $A$ 中没有主元素。
        假设 $A$ 中的 $n$ 个元素保存在一个一维数组中,请设计一个尽可能高效的算法,找出 $A$ 的主元素。
        若存在主元素,则输出该元素;否则输出 $-1$。要求:
        \begin{subqitems}
            \subqitem 给出算法的基本设计思想。
            \subqitem 根据设计思想,采用 C 或 C++或 Java 语言描述算法,关键之处给出注释。
            \subqitem 说明你所设计算法的时间复杂度和空间复杂度。
        \end{subqitems}
    \end{bbox}

    \begin{bbox}
        \qitem 【2018 统考真题】给定一个含 $n$ ($n\geqslant 1$) 个整数的数组,请设计一个在时间上尽可能高
        效的算法,找出数组中未出现的最小正整数。例如,数组 $\{-5,3,2,3\}$ 中未出现的最小正
        整数是 $1$; 数组 $\{1,2,3\}$ 中未出现的最小正整数是 $4$。要求:
        \begin{subqitems}
            \subqitem 给出算法的基本设计思想。
            \subqitem 根据设计思想,采用 C 或 C++语言描述算法,关键之处给出注释。
            \subqitem 说明你所设计算法的时间复杂度和空间复杂度。
        \end{subqitems}
    \end{bbox}

    \begin{bbox}
        \qitem 【2020 统考真题】定义三元组 $(a,b,c)$ ($a,b,c$ 均为整数)的距离 $D=|a-b|+|b-c|+|c-a|$。
        给定 $3$ 个非空整数集 $S_1, S_2$ 和 $S_3$,按升序分别存储在 $3$ 个数组中。请设计一个尽可能
        高效的算法,计算并输出所有可能的三元组 $(a,b,c)$ ($a\in S_1, b\in S_2, c\in S_3$) 中的最小距
        离。例如 $S_1=\{-1,0,9\}, S_2=\{2,5,-10,10,11\}, S_3=\{2,9,17,30,41\}$, 则最小距离为 $2$,
        相应的三元组为 $(9,10,9)$。要求:
        \begin{subqitems}
            \subqitem 给出算法的基本设计思想。
            \subqitem 根据设计思想,采用 C 语言或 C++语言描述算法,关键之处给出注释。
            \subqitem 说明你所设计算法的时间复杂度和空间复杂度。
        \end{subqitems}
    \end{bbox}

\end{qitems}
