\subsection{交换排序}\qanswerloc{355}

\begin{qitems}
    \begin{bbox}
        \qitem 已知线性表按顺序存储,且每个元素都是不相同的整数型元素,
        设计把所有奇数移动到所有偶数前边的算法(要求时间最短,辅助空间最小)。
    \end{bbox}
    \begin{bbox}
        \qitem 试编写一个算法,使之能够在数组 \lstinline{L[1...n]} 中找出第 $k$ 小的元素
        (从小到大排序后处于第 $k$ 个位置的元素)。
    \end{bbox}
    \begin{bbox}
        \qitem 荷兰国旗问题:设有一个仅由红、白、蓝三种颜色的条块组成的条块序列,
        存储在一个顺序表中,请编写一个时间复杂度为 $O(n)$ 的算法,使得这些条块按红、白、蓝的顺序排好,
        即排成荷兰国旗图案。请完成算法实现:
        \quad \lstinline|typedef enum{RED, WHITE, BLUE} color; //设置枚举数组|
        
        \quad \lstinline|void Flag_Arrange(color a[],int n) { ... }|
    \end{bbox}
    \begin{bbox}
        \qitem 【2016 统考真题】已知由 $n$ ($n \ge 2$) 个正整数构成的集合 $A=\{a_k | 0 \le k < n\}$,将其划分为两个不相交的子集 $A_1$ 和 $A_2$,元素个数分别是 $n_1$ 和 $n_2$,$A_1$ 和 $A_2$ 中的元素之和分别为 $S_1$ 和 $S_2$。设计一个尽可能高效的划分算法,满足 $|n_1-n_2|$ 最小且 $|S_1-S_2|$ 最大。要求:
        \begin{subqitems}
            \subqitem 给出算法的基本设计思想。
            \subqitem 根据设计思想,采用 C 或 C++语言描述算法,关键之处给出注释。
            \subqitem 说明所设计算法的平均时间复杂度和空间复杂度。
        \end{subqitems}
    \end{bbox}
\end{qitems} 